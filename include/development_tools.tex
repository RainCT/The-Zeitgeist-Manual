\section{Languages used}

\subsection{Zeitgeist Engine}
The Zeitgeist engine is primarily written in Python and minimum supported version is Python 2.7

\subsection{Zeitgeist Datahub}


\subsection{Libraries and Datasources}
\subsection{Why Python for Engine?}
GNOME Zeitgeist started from the Gimmie/Mayanna codebase that was written 
in python. The tradition continued. Although we had several thoughts to move 
to Mono, C and Vala it never happened for several reasons. 
\begin{itemize}
\item Mono would have created a big debate in the community since a lot of 
GNOME hackers dislike it, however at the very beginning all ZG developers were 
keen with mono, yet to ensure no political issues we decided to stay with python.
\item C was and still is an issue since, only a few or the core devs are familiar 
with C and we will lose momentum.
\item Vala would have made Zeitgeist more a GNOME project than a desktop project 
and there is not much gain now in moving to any other programing language since 
speed and memory consumption wont be affected heavily in comparison to the effort 
we will have to put in it. That time Vala wasn't an option either because it 
doesn't support some D-Bus features we needed.
\end{itemize}




\section{Other technologies}
\subsection{SQLite}
\subsection{DBus}
\subsection{GLib}

\section{Bazaar}
Bazaar is a distributed version control system. Using a distributed version control 
system like bazaar developers can collaborate in a completely distributed manner 
without having to know what code changes the other developer is doing. Every developer 
has a complete copy of the source code with all the change history.

To collaborate amongst the developers there is a centralized server where all the 
developers can push code and pull changes made by others. Let us imagine a scenario 
in which various developers have to collaborate.
\begin{enumerate}
\item The central server contains the source code.
\item Three developers make a clone of the source code using bazaar.
\item All three of them go to some place without an internet connection.
\item All of them keep making changes to the codebase.
\item Developer A pushes his code on the server.
\item Rest two developers pull from the server and get Developer A's changes.
\item Developer B pushes his changes.
\item Developer A and C pull the changes.
\item Developer A has B's changes and Developer B has Developer A's changes.
\item Developer C has changes from all the two developers plus his own.
\item Developer C pushes the changes on the server.
\item Developer A and B pull th changes from the server.
\item Now all the three developers have each other changes and server has changes 
from all three of them.
\end{enumerate}

Bazaar also maintains history of the changes. It can be told to keep a watch on the 
files specified and it can then tell which files have changes and what has changed. It 
can even tell the contents which has changes. After making some changes we can commit the 
changes. Each commit is like a \textbf{milestone}. All the changes made after the commit are 
called changes, which can be then committed. Anytime a person can see the changes 
made between the various commits. 

It is advisable to have one logical unit of work in a commit instead of making 10,000 
lines of changes in a single commit. Example changes made to add a new entry in the toolbar 
can qualify as a commit. If the changes are very big, then this should be broken down into 
smaller and more managable changes. 

There are many reasons for keeping the changes in a single commit small and managable: 

\begin{itemize}
\item The changes made between commits can be viewed by a diff tool like \textit{kdiff} and \textit{meld}. 
If changes are small, reading and understanding the changes is easier.
\item When changes are small, it is easy to find out the commit which introduced a regression.
\item Many times code is submitted for review before it is accepted. Having a huge 8000 lines 
of changes makes it very difficult for the maintainer of the project to understand.
\end{itemize}

\section{Launchpad}
\section{Internet Relay Chat}
\section{Mailing lists}