\section{Introduction}

As mentioned before Zeitgeist is an event log and not a search engine or 
file tracker. It just logs the various events occouring on the system 
like file opened, file modified, call placed, IM received etc.

\section{Engine}

At the heart zeitgeist consists of an engine which runs as a daemon. It's 
work is to simply receive events, store it in the database, provide the 
events when requested for. The daemon exposes a per-user D-Bus which runs 
as a SessionBus.


\noindent
\linebreak 
The daemon stores all the events and related metadata in an sqlite database 
which is stored in 

\texttt{\$XDG\_DATA\_HOME/zeitgeist/activity.sqlite}

\noindent
where \texttt{\$XDG\_DATA\_HOME} equals to \texttt{\$HOME/.local/share} in 
most of the cases.


\noindent
\linebreak 
If there are more than one user on the system, then every user will have 
their own Zeitgeist SessionBus and thus their own seperate sqlite database. 
This behaviour of SessionBus is useful for keeping activities of different 
users stored under their own directory.

\section{The Basic Unit: Event}

If the engine is the heart, then the role of blood is being taken up by Events. 
Events are packet of information which contain the data about any kind of activity 
occouring on the system.

An event contains:
\begin{itemize}
\item \textbf{id} - Every event has a unique identifier which can be used to identify 
an event from others. It is of type int
\item \textbf{timestamp} - This is the time when the event occoured. The value stored is 
the number of milliseconds since UNIX epoch. UNIX epoch is the moment which is 
00:00:00 UTC on Thursday, January 1st, 1970.
\item \textbf{interpreation} - This defines "what happened" using a formal URI. The 
predefined set of URIs are available and will be covered later.
\item \textbf{manifestation} - This defines “how did this happen” using a formal URI. 
The predefined set of URIs are available and will be covered later.
\item \textbf{actor} - The URI defining the entity responsible for this event. 
In most of the case it is an application. In case of application the 
application:// URI is used. e.g. application://firefox.desktop
\item \textbf{subjects} - This just represents the subject of the event. There can be more 
than one subjects but in majority of the cases, only one is needed
\item \textbf{payload} - An array of bytes which can be used for freeform storage of any 
data associated with the event. The encoding or format of the data is not handled 
by the engine and is left upon the clients.
\end{itemize}