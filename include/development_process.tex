\section{How we handle bugs}
\section{How we handle blueprints}
\section{How we handle merge requests}
\section{How we test our work}
\section{How we manage releases}
A release is discussed between the various module maintainers. 
After the discussion the goals for the upcoming release is set 
and a release date is decided. This release date date is a vague 
estimation and can change depending on various factors like 
feature set, number of bugs, manpower availability etc. 
Depending on the feature sets and bugs as well as the personal 
availability a release manager is assigned to take the lead 
on driving the release.

There are a few things to be kept in mind:
\begin{itemize}
\item Most of the releases are taken care by a single release manager 
with rest of the team members assisting him. 
\item The selection of release manager has no fixed criteria. The 
person who does the most amount of work in a particular cycle takes 
up the role.
\item Release manager is not a status symbol, it is a responsibility.
\item We usually switch roles from developer to tester to 
release manager to keep everyone refreshed.
\item Switching roles and having different release manager for releases 
keeps us from getting burnt out.
\item Every release has a name. The release manager is the person who 
picks up the name. He can consult other team members for suggestions.
\end{itemize}

\section{Description of our development process}
\section{Governance model}